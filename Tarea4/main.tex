\documentclass[letterpaper,global 12pt]{article}
\usepackage[utf8]{inputenc}
\usepackage{amsmath}
\usepackage{amssymb}
\usepackage{latexsym}
\usepackage[svgnames,x11names]{xcolor}
\usepackage{dsfont}
\usepackage{amsfonts}
\usepackage{array}
\usepackage{multirow}
\usepackage{fancyhdr}
\pagestyle{fancy}
\title{Para dar el formulaso }
\author{César Eduardo Rosete}
\date{Octubre 2022}
\begin{document}
\lhead{\leftmark}
\chead{César Eduardo Rosete Gómez}
\rhead{\thepage}
\lfoot{\leftmark}
\cfoot{César Eduardo Rosete Gómez}
\rfoot{\thepage}
\maketitle
\definecolor{vision}{RGB}{7,70,4}
\section{Cálculo}
\begin{itemize}
    \item[\Large\ddagger] "La chicharonera" es una fórmula que nos permite obtener los valores para x en una ecuación de segundo grado. Este tipo de fórmula solo se puede obtener para ecuaciones de hasta quinto grado. 
    \begin{equation*}
        x=\frac{-b\pm\sqrt{b^2-4ac}}{2a}
    \end{equation*}\color{magenta}
    \item[\mho]La suma de Gauss nos da la suma de una serie de numeros naturales. 
    \begin{equation*}
        \sum_{i=1}^{n}i=\frac{n(n+1)}{2}
    \end{equation*}\color{black}
    \item [\Large\dagger] Coloquialmente conocida como la fómula de la vaca, es una fórmula que permite integrar cosa como los logaritmos. Es considerada como uno de los metodos básicos de integración.
    \begin{equation*}
    \int udv=uv-\int vdu
    \end{equation*}
    \item[\clubsuit] El teorema de Pitágoras es uno de los más usados en matemáticas. Implica que la hipotenusa de un triángulo rectángulo es igual a la raíz del cuadrado de sus catetos. 
    \begin{equation*}
        a^2=b^2+c^2
    \end{equation*}
    \item[\heartsuit] La definición formal de la derivada nos permite estudiarla como un límite y analizar puntos en los que se puede presentar una discontinuidad. 
    \begin{equation*}
        f'(x)=\lim_{h\to 0}\frac{f(h+x)-f(x)}{h}
    \end{equation*}
\end{itemize}
\section{Física}
\begin{itemize}\color{magenta}
    \item [\vdash] Fuerza igual a masa por aceleración es una de las fórmulas clásicas de la física. Se le utiliza para los modelos newtonianos.
    \begin{equation*}
        \Vec{F}=m\Vec{a}
    \end{equation*}\color{black}
    \item [$\alpha$] La fuerza de gravitación mide las fuerzas atracción entre dos cuerpos de masa $m_1$ y $m_2$
    \begin{equation*}
        \Vec{F}=G\frac{m_1*m_2}{r^2}
    \end{equation*}
    \item[$\omega$]La velocidad se define como la relación entre distancia por unidad de tiempo. Esto se ve como la derivada(una tasa de cambio) de la distancia en función del tiempo. 
    \begin{equation*}
        \Vec{v}=\frac{dx}{dt}
    \end{equation*}
    \item[$\gamma$]De forma similar a $\omega$ la aceleración ($\Vec{a}$) se define como el cambio de la velocidad por unidad de tiempo. 
    \begin{equation*}
        \Vec{a}=\frac{d\Vec{v}}{dt}
    \end{equation*}
    \item[\flat] Al analizar el movimiento unidireccional llegamos a poner a la distancia, ya sea vertical u horizontal en función de la velocidad y el tiempo. 
    \begin{align*}
        \Delta x= \Vec{v_i}t+\frac{1}{2}\Vec{a}*t^2\\
        \Vec{v_f}^2=\Vec{v_i}^2+2\Vec{a}d\\
    \end{align*}
    \item[\natural] La ley de Hooke nos dice la relación entre una compresión o estiramiento y la fuerza que se ocupa para lograrlo en un resorte. Estas se relacionan mediante una constante k. 
    \begin{equation*}
        \Vec{F}=k\Delta x
    \end{equation*}
    \item[\sharp] La energía de un sistema resulta de la suma de energía cínetica y la energía potencial.
    \begin{align*}
        E_T&=E_c+E_p\\
        E_c&=\frac{1}{2}m\Vec{v}^2\\
        E_p&=mgh\\
    \end{align*}
    La fórmula de la energía potencial suele variar en relación al sistema. 
    \item[\spadesuit] El trabajo es un vector que mide la relación fuerza distancia.
    \begin{equation*}
        \Vec{W}=\Vec{F}*d
    \end{equation*}
    \item[\dagger] La fuerza entre cargas se describe a partir de la ley de coulmb.
    \begin{equation*}
        \Vec{F}=K\frac{q_1*q_2}{r^2}
    \end{equation*}
    \item[\Re]La ley de Ohm nos da la relación de proporcionalidad entre la intensidad de corriente eléctrica que circula por un conductor en relación al voltaje y la restistencia. 
    \begin{equation}
        I=\frac{V}{R}
    \end{equation}
    \item[\aleph]La potencia eléctrica es la cantidad de energía que consume un dispositivo eléctrico por unidad de tiempo, y esta dado por:
    \begin{equation}
        P=V*I
    \end{equation}
    \item[\Im] De 1 y 2 se obtiene:
    \begin{align*}
        P&=I^2*R\\
        P&=\frac{V^2}{R}
    \end{align*}
    \item[\diamondsuit]La intensidad del campo magnético \textbf{B} producido por un selenoide de \textbf{\textit{N}} vueltas y de longitud \textbf{\textit{L}}, en el que circula una intensidad de corriente \textbf{\textit{I}} se obtiene de:
    \begin{equation*}
        B=\frac{N*\mu*I}{L}
    \end{equation*}
    \item[\vdash]La intensidad del campo magnético \textbf{B} inducido por un conductor recto, por el que circula una intensidad de corriente \textbf{\textit{I}} a una distancia \textbf{\textit{d}} del conductor se obtiene de:
    \begin{equation*}
        B=\frac{\mu *I}{2\pi *d}
    \end{equation*}
    \item[$\psi$]Una espira se obtiene de doblar en forma circular un conductor recto. La intensida de corriente \textbf{B} producida por la espira de radio \textbf{\textit{r}} por la que circula una corriente eléctrica \textbf{\textit{I}} esta dada por:
    \begin{equation*}
        B=\frac{\mu *I}{2r}
    \end{equation*}
    \item[$\rho$]La ecuación de Bernoulli nos dice que dado un fluido cuyo flujo sea estacionario, las sumas de sus energías cinetica, potencial y de presión son iguales en un punto a como en un punto b. 
    \begin{align*}
        E_c_a+E_p_a+E_presión_a&=E_c_b+E_p_b+E_presión_b\\
        \frac{\Vec{v}_a^2}{2}+gh_a+\frac{P_a}{\rho}&=\frac{\Vec{v}_b^2}{2}+gh_b+\frac{P_b}{\rho}
    \end{align*}
    \item[$\pi$]La ley de Snell nos da la relación entre el ángulo de incidencia y el ángulo de refracción,así como los índices de refracción.
    \begin{equation*}
        n_1\sin\theta _1=n_2\sin\theta _2
    \end{equation*}
\end{itemize}
\section{Contaduría}
\begin{itemize}\color{magenta}
    \item [\wp]El interés compuesto el interés de un capital al que se van acumulando sus créditos o intereses para que produzcan otros
  \begin{equation*}
        C_f=C_i(1+\frac{r}{100})^t
    \end{equation*}\color{black}
\end{itemize}
\end{document}
