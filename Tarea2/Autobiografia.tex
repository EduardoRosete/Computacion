\documentclass[letterpaper,12pt]{article}
\usepackage[utf8]{inputenc}

\title{Dos bases. ¿Será hit y doble carrera?}
\author{César Eduardo Rosete Gómez}
\date{11 septiembre 2022}

\begin{document}

\maketitle

\section{Academia}
\subsection{Mi pasado}
{\large{Mi educación tanto primaria, secundaria y preparatoria fueron completados en la Escuela Moderna Americana.}} \tiny{Resulto ser que a los 4 años de edad, yo ya contaba con una necedad implacable y en el momento de que se me pregunto de las opciones que se me habían presentado elegí esa. En la preparatoria se me ofreció el cambio a una prepa UNAM, sin embargo, la fama de mi escuela recae en esos tres últimos años por lo que decidí quedarme.}\\
\large{Durante la mayoría de la primaria y buena parte de la secundaria viví cerca de la salida Cuernavaca por lo que mi escuela situada en Coyoacán me quedaba retirada. Por tal motivo me llevaba mi papá en su carro. Por obvias razones el único camino era aquel que te lleva por Insurgentes y cruzaba por la UNAM para llegar más rápido.} \tiny{Ahí empezó casi inconscientemente un gusto por esta universidad.}\\
\large{En la siguiente mitad de mi estancia en esa escuela me mude a las cercanías por lo que únicamente caminaba alrededor de 15 minuntos para llegar.} \tiny{Curiosamente en ese momento fue cuando empecé a presentar mayor cantidad de retrasos.}
\subsection{Actualidad }
\large{La UNAM se encuentra a 15 minutos en carro de mi casa, claro que terminan siendo 20 a 30 minutos por cuestiones de tráfico.} \tiny{Podría caminar, de hecho, en algún momento lo hice pues estudie francés en el Centro Cultural Universitario. Sin embargo, el agregar la caminada hasta ciencias no lo hace conveniente en especial porque me ocupo en otra carrera en las mañanas. Debido a esto, mi ruta cambia bastante.} \large{Debo cruzar tanto avenida Revolución como avenida Insurgentes, lo cual agrega alrededor de 10 minutos a mi ruta, si bien me va, pues suelo tomarla en horas pico. Finalmente, para llegar a la facultad tomo avenida Insurgentes y el circuito interno de la UNAM para llegar.} 
\subsection{¿Porque física?}
\tiny{Elegí física casi en el último año de preparatoria, no sabía realmente que quería.} \large{Pero en ese año tuve un profesor que me hizo enamorarme de la materia.} \tiny{Su forma particular de dar clases revivió una curiosidad y una forma de pensar el mundo que habia perdido gracias a otros profesores. Ya que a pesar de que me toco en pandemia, el hizo el mayor esfuerzo posible para que su clase fuera didáctica a pesar de que él está casi ciego y es bastante mayor. Esto lo logró, manteniendo los laboratorios que dieron como resultado las prácticas con valores experimentales más raros en un buen tiempo. También, fomento la colaboración entre alumnos y siempre busco el mayor crecimiento y aprendizaje que se pudiera obtener en sus alumnos.} 
\section{Pasatiempos}
\subsection{Historia}
\large{Desde una temprana edad he tenido una intensa curiosidad por la historia, por lo que ha pasado y como se vivía en otra época. En mis tiempos libres, que cada vez son más cortos, suelo estudiar historia} \tiny{En algún momento si tenía una predilección por épocas o por temas específicos. Actualmente me importa mucho menos los aspectos grandiosos (en términos de la importancia que se les da) como lo serían batallas, guerras, personajes, ... y me intereso mucho más el día a día de las personas. De igual manera, he incluido en los últimos años nuevas perspectivas y una visión más crítica de lo que investigo lo que ha cambiado fundamentalmente en lo que me enfoco.} 
\subsection{Objetos antiguos}
\large{De manera cercana a mi interés de la historia tengo un interés por investigar los objetos viejos.} \tiny{Desde mi perspectiva son las evidencias de como se pensaba en el pasado. De igual forma, nos dicen mucho de cómo se vivía en el pasado, de las preocupaciones de la gente y como se visualizaban soluciones para dichos problemas. Por ejemplo, la maquina de escribir es un mecanismo sumamente complejo, que requiere de atención y ajustes. De cierta forma para mucha gente este dispositivo fue el precursor a la computadora y le dan el mismo uso que esta. Pero una máquina de escribir no solo dice eso, sino que también cuenta sobre conspiraciones y el poder del dinero ya que Olivetti uno de los más famosos productores de máquinas de escribir estuvo a punto de empezar a producir ordenares personales, sin embargo, fue asesinado un año antes en circunstancias turbias.}
\section{Música}
\subsection{Folk}
\large{El folk es lo que se considera como música popular, pero de cierta forma se apega a la cultura de un determinado grupo.} \tiny{Ya sea que se apegue al estilo americano, por esto me refiero a lo que se tocaba tradicionalmente, o al aleman o al mexicano, etc. En el caso de América Latina por la ascendencia española es usual que esto incluya instrumentación como guitarra.} \large{Es lo que más suelo escuchar, realmente no me apego a este genero por un motivo en particular. 
En este genero en el estilo americano, mi artista favorito es Jackson C. Frank y una de mis canciones favoritas de él es {\it{Marlene}}. Por otro lado,  {\it{The Dublineers}} son una banda irlandesa de los 60's y una de sus canciones más famosas es {\it{On the Rocky Road to Dublin}}.} 
\subsection{Clásica}
\large{Desde muy pequeño se me inculco el gusto por la música clásica.} \tiny{De hecho llegue a tocar el violin por varios años e incluso participe en ensables escolares.} \large{Suelo escuchar este tipo de música para concentrarme o no distraerme más.} \tiny{En este genero sin duda hay miles de compositores, pero también de interpretes y esto para mi al menos tiene un gran efecto en como recibe uno la pieza que escucha.} \large{Mi interprete favorita en el momento es Anna Göckel, interpretando la Sonnata no.1 en G menor de Bach y {\it{Sei Solo}} de Bach igualmente. En una nota más moderna {\it{Slowland}} de Lambert es otra pieza que puede llamar la atención pues es de este año.}
\section{¿Qué onda con el nombre?}
\large{Dos bases, es una pequeña analogía con uno de mis deportes favoritos de donde siento que estoy parado.} \tiny{En el baseball se tienen tres bases y un home; el punto es recorer las bases con el tiempo que se te otorga al momento que batea otro miembro de tu equipo. Si hay hit puedes correr e intentar que no te ponchen. Si no tu otra opción es robar la base. Las segunda y tercera bases se llenaron con la secundaria y preparatoria, por lo menos para esta entrada. En este momento siento que se espera un hit de cuatro a cinco años para dar tiempo al corredor en segunda y tercera de recorrer y anotar literalmente dos carreras. No tengo la más remota idea de si se lograra o no. En cualquier caso, solo hay un out en la caja registradora (la anotación), por lo que hay chance para otro bateador.\\
Claro que la analogía tiene el ligerisimo error de que para que se acabe una entrada te deben ponchar tres veces. A mi al menos me lleva a preguntarme ¿qué será eso y si equivale a ganar algo o no? Pero en la vida creo que no sé gana o pierde, solo se avanza como en el baseball se cambia de entradas.}
\end{document}
